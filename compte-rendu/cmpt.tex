\documentclass[12pt]{article}
\usepackage[french]{babel}
\usepackage[T1]{fontenc}
\usepackage{graphicx} % Required for inserting images
\usepackage[letterpaper,margin=3cm]{geometry}
\usepackage[export]{adjustbox}
\usepackage{listings}
\usepackage{tabularx}
\usepackage{multirow}
\usepackage[table]{xcolor}
\usepackage{svg}
\usepackage{tocloft}
\usepackage{algorithmic}
\usepackage{indentfirst}
\usepackage{hyperref}
\usepackage{array}
\usepackage{float}

\renewcommand{\thesection}{\Roman{section}} 
\renewcommand\thesubsection{\arabic{subsection}}

\setlength{\cftsecnumwidth}{3em} 
% ajuste la largeur de la colonne des numéros de section

\setlength{\parindent}{1cm}
\sloppy


%--- begin document ------------------------------------------------------------

\title{Projet de raisonnement propositionelle}
\author{Edouard.H Théo.R.V}
\date{2022--2023}

\begin{document}
    \begin{figure}
        \includegraphics[scale=0.3, right]{logo-univ-rouen-normandie-noir.png}
    \end{figure}

    \maketitle

    \begin{abstract}
        Ce document constitue notre compte rendu du projet de raisonnement
        propositionelle traitent d'une implémentation de la méhtode des 
        tableaux et de fonction de test. Il s'agit d'une méthode permettant de 
        déterminer la satisfaisabiliter d'une formule booléen. Les fonctions de 
        test permette de vérifié empiriquement l'implémentation de la méthode.  
        
        Dans un premier temps, nous présenterons notre implémentation de la 
        méthode des tableaux et des divers choix présent dans celle-ci. Ensuite,
        nous présenterons nos fonctions de test. Pour finir, nous aborderons les
        difficulters liés à la production de ce projet.
    \end{abstract}

    \newpage

    \tableofcontents

    \newpage

    \section{Implémentation de la méthode des tableaux}
    \section{Les fonctions de teste}
    \section{Les difficulter rencontré}
    \section{Conclusion}
\end{document}