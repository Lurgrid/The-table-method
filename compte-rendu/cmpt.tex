\documentclass[12pt]{article}
\usepackage[french]{babel}
\usepackage[T1]{fontenc}
\usepackage{graphicx} % Required for inserting images
\usepackage[letterpaper,margin=3cm]{geometry}
\usepackage[export]{adjustbox}
\usepackage{listings}
\usepackage{tabularx}
\usepackage{multirow}
\usepackage[table]{xcolor}
\usepackage{svg}
\usepackage{tocloft}
\usepackage{algorithmic}
\usepackage{indentfirst}
\usepackage{hyperref}
\usepackage{array}
\usepackage{float}
\usepackage{caption}

\renewcommand{\thesection}{\Roman{section}} 
\renewcommand\thesubsection{\arabic{subsection}}

\setlength{\cftsecnumwidth}{3em} 
% ajuste la largeur de la colonne des numéros de section

\setlength{\parindent}{1cm}
\sloppy


%--- begin document ------------------------------------------------------------

\title{Projet de raisonnement propositionnel}
\author{Edouard.H Théo.R.V}
\date{2022--2023}

\begin{document}
    \begin{figure}
        \includegraphics[scale=0.3, right]{logo-univ-rouen-normandie-noir.png}
    \end{figure}

    \maketitle

    \begin{abstract}
        Ce document constitue notre compte rendu du projet de raisonnement
        propositionnel traitent d'une implémentation de la méthode des 
        tableaux et de fonction de test. Il s'agit d'une méthode permettant de 
        déterminer la satisfesabiliter d'une formule booléenne. Les fonctions de 
        test permettent de vérifier empiriquement l'implémentation de la méthode.  
        
        Dans un premier temps, nous présenterons notre implémentation de la 
        méthode des tableaux et des divers choix présent dans celle-ci. Pour 
        finir, nous aborderons les difficultés liées à la production de ce 
        projet.
    \end{abstract}

    \newpage

    \tableofcontents

    \newpage

    \section{Implémentation de la méthode des tableaux}

    \subsection{Formule}

    Tout d'abord, nous avons dès le début de la conception de ce projet mis en 
    place les opérateurs étendus vus en TD.\@ On peut notamment citée les 
    opérateurs, \textbf{xor}, \textbf{nor}, \textbf{nand}, \textbf{diff} et 
    \textbf{equiv} de symbole respectif $\oplus$, $\uparrow$, $\downarrow$, 
    $\setminus$ et $\Leftrightarrow$. 

    La fonction \textbf{string\_of\_formule} est une fonction récursive non 
    terminale, ce choix est dû à plusieurs raisons. La première étant que sa 
    complexité espace sera logarithmique par la division de la formule par deux
    a chaque appel récursif. La deuxième raison réside dans l'hypothèse que les
    formule afficher ne devrais pas être d'une taille démesurée enfaite une 
    formule trop grande ne pourrait pas être affichée sur un terminal de taille
    standard. De plus ce choix favorise la compréhension de cette fonction.

    En ce qui concerne la fonction \textbf{eval} ayant déjà implémenté la
    partie de `base' en TP tout le travail, c'est focaliser sur les opérateurs
    étendus. Ces opérations n'étant pas implémenté de façon standard en Ocaml,
    nous avons donc à partir des tables de vérités mises en place des formules 
    équivalente permettant alors leur implémentation, voir 
    table~\ref{tab:tab-verite} pour les formules équivalente résultant des 
    tables de vérités.  

    \begin{table}[H]
        \centering
        \hfill
        $\begin{array}{c c c}
            \hline
            a & b & a \oplus b\\
            \hline
            0 & 0 & 0\\
            0 & 1 & 1\\
            1 & 0 & 1\\
            1 & 1 & 0\\
            \hline
            \multicolumn{3}{c}{(a \setminus b) \lor (b \setminus a)}
        \end{array}
        \hfill
        \begin{array}{c c c}
            \hline
            a & b & a \uparrow b\\
            \hline
            0 & 0 & 1\\
            0 & 1 & 1\\
            1 & 0 & 1\\
            1 & 1 & 0\\
            \hline
            \multicolumn{3}{c}{\lnot (a \land b)}
        \end{array}
        \hfill
        \begin{array}{c c c}
            \hline
            a & b & a \downarrow b\\
            \hline
            0 & 0 & 1\\
            0 & 1 & 0\\
            1 & 0 & 0\\
            1 & 1 & 0\\
            \hline
            \multicolumn{3}{c}{\lnot (a \lor b)}
        \end{array}
        \hfill
        \begin{array}{c c c}
            \hline
            a & b & a \setminus b\\
            \hline
            0 & 0 & 0\\
            0 & 1 & 1\\
            1 & 0 & 1\\
            1 & 1 & 0\\
            \hline
            \multicolumn{3}{c}{a \lor b}
        \end{array}
        \hfill
        \begin{array}{c c c}
            \hline
            a & b & a \Leftrightarrow b\\
            \hline
            0 & 0 & 0\\
            0 & 1 & 1\\
            1 & 0 & 1\\
            1 & 1 & 0\\
            \hline
            \multicolumn{3}{c}{(a \to b) \land (b \to a)}
        \end{array}$
        \hfill
        \captionsetup{position=bottom}
        \caption{Tables de vérités et formule équivalente.}\label{tab:tab-verite} 
    \end{table}

    \subsection{RandomFormule}

    Pour la fonction \textbf{random\_form}, nous avons fais le choix de la 
    divisé en plusieur sous-fonction. Un tel choix permet une compréhension plus
    aisé du traitement effectué dans la fonction. Les quatres fonctions 
    \textbf{random\_atome}, \textbf{random\_n\_operator}, 
    \textbf{random\_u\_operator} et \textbf{random\_b\_operator} correponde 
    chacune à l'obtention d'un élément de façon pseduo-aléatoire, un atome, un 
    opérateurs nullaire, unaire et binaire respectivement au fonction sité plus 
    haut. Une telle division permet aussi de facilité un possible futur ajout 
    d'opérateurs, d'où la création d'une fonction pour obtenir un opérateur 
    unaire de façon alétoire bien qu'il n'aille qu'un seul. Toutes les autres 
    fonction obtients l'element cible en tirant un nombre pseudo-aléatoire à 
    l'aide du module \textbf{Random}.
    
    \subsection{}

    \subsection{Test}

    La fonction \textbf{to\_alea\_inter}, a été implémentée à l'aide d'une 
    fonction auxiliaire qui sera renvoyer à l'utilisateur partiellement 
    exécuter (sans tous ces paramètres, pour renvoyer une fonction de type 
    interprétation).

    \section{Les difficultés rencontrées}
    
    \section{Conclusion}

\end{document}